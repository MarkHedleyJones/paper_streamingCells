\documentclass[10pt,final,journal]{IEEEtran}
\usepackage{amsfonts}
\usepackage{amssymb}
\usepackage{graphicx}

\title{Feasibility of Harvesting Power To Run A Domestic Water Meter Using Streaming Cell Technology}

\author{Mark~H.~Jones, and Prof.~Jonathan~Scott}

\begin{document}
    \maketitle

    \begin{abstract}
        We investigate the possibility of using streaming cells as a means of harvesting energy from potable water.
        We fabricate a number of crude streaming potential cells, each with differing geometry.
        We measure the electrical power developed from each of these cells over a range of pressure differentials.
        We estimate the amount of energy required to operate a simple data logger and a typical domestic water flow profile.
        Based on this data we calculate the size of a streaming cell harvester required to log domestic water consumption.
    \end{abstract}

    \section{Introduction}
    Domestic and commercial water metering is becoming increasingly important throughout the world as water.
    There are two methods available to water utility companies when recording the volume of water delivered to consumers.
    \begin{enumerate}
        \item \emph{Mechanical readout:}
            This involves having a human sight the meter's readout from time to time.
            Generally the reading frequency is kept to a minimum and estimate readings are made in the interim.
        \item \emph{Electronic telemetry:}
            Wireless data transmission commonly used as the method of data transport.
            Due to the location of water metering, electrical power is provided by long life batteries.
            These batteries are usually non-rechargeable and therefore must be replaced once exhausted.
        \end{enumerate}
    Having a cheap and reliable way to retrieve metering information is the most important factor for utility companies.
    The methods previously mentioned are costly to the utility as they require periodic human intervention.
    The possibility of harvesting energy at the meter will allow periodic data transmission with less maintenance.
    Removing batteries from electrical meters also reduces unwanted electrical waste.

    Streaming cells are commonly used in membrane science to determine the zeta potential of a membrane surface.
    This is an important factor in the design and operation of membrane based processes. \cite{Daiguji2004b}
    Streaming cells have also been investigated as a means of pumping water and generating power.\cite{Daiguji2004b, Davidson2008, Hon2004, Olthuis2005}

    In \cite{CherngHon2012}, salinity gradients are shown to produce electrical power 

    \bibliographystyle{plain}
    \bibliography{library} 

%\begin{thebibliography}{99}
%
%    \bibitem{CherngHon2012}
%    Kar~Cherng~Hon, Cunlu~Zhao, Chun~Yang, and Seow~Chay Low,
%    ``A method of producing electrokinetic power through forward osmosis'',
%    Appl. Phys. Lett. 101, 143902 (2012); doi: 10.1063/1.4756903
%
%    \bibitem{Daiguji2004b}
%    Hirofumi~Daiguji, Peidong~Yang, Andrew~Szeri, and Arun~Majumdar,
%    ``Electrochemomechanical Energy Conversion in Nanofluidic Channels'',
%    Nano Letters, 2004, vol. 4, 12, pp2315-2321
%
%    \bibitem{Davidson2008}
%    Christian~Davidson, Xiangchun~Xuan,
%    ``Effects of Stern layer conductance on electrokinetic energy conversion in nanofluidic channels'',
%    Electrophoresis-NEEDS-CHECKING
%
%    \bititem{Hon2012}
%    K.C.~Hon, C.~Zhao, C.~Yang, and S.C.~Low,
%    ``A method of producing electrokinetic power through forward osmosis'',
%    referenceTODO
%
%
%    \bibitem{Olthuis2005}
%    Authors,
%    ``Energy from streaming current and potential'',
%    journal
%
%    \bibitem{XuanLi2006}
%    Authors,
%    ``Thermodynamic analysis of electrkinetic energy converson'',
%    journal
%
%    \bibitem{RenStein2008}
%    Authors,
%    ``Slip-enhanced electrokinetic energy conversion in nanofluidic channels'',
%    journal
%
%    \bibitem{vanderHeyden2006}
%    Authors,
%    ``Electrokinetic energy conversion efficiency in nanofluidic channels'',
%    journal
%
%    \bibitem{DavidsonXuan2008b}
%    Authors,
%    ``Electrokinetic energy conversion in slip nanochannels'',
%    journal
%
%    \bibitem{Yang2003}
%    Authors,
%    ``Electrokinetic microchannel battery by means of electrokinetic and microfluidic phenomena'',
%    journal
%
%    \bibitem{vanderHeyden2007}
%    Authors,
%    ``Power Generation by Pressure-Driven Transport of Ions in Nanofluidic Channels'',
%    journal
%
%    \bibitem{Xie2008}
%    Authors,
%    ``Electric energy generation in single track-etched nanopores'',
%    journal
%
%    \bibitem{Daiguji2006}
%    Authors,
%    ``Theoretical studi on the efficiency of nanofluidic batteries
%
%
%\end{thebibliography}



\end{document}
