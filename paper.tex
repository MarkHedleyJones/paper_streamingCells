\documentclass[10pt,final,journal]{IEEEtran}
\usepackage{amsfonts}
\usepackage{amssymb}
\usepackage{graphicx}
\usepackage{siunitx}
% \usepackage[dvipsnames]{color}

\title{Feasibility of Harvesting Power To Run A Domestic Water Meter Using Streaming Cell Technology}

\author{Mark~H.~Jones, and Jonathan~Scott}

\begin{document}
    \maketitle

    \begin{abstract}
        We investigate the possibility of using streaming cells as a means of harvesting energy from a town water supply.
        We verify streaming cell measurements found in the literature and extend those measurements over a wider range of dimensions.
        We measure the electrical power developed from ten streaming cells over a range of pressure differentials using town water.
        We estimate the amount of energy required to operate a simple data logger and transmitter.
        We estimate the amount of energy available from a typical domestic household based on water usage data.
        From these estimates we calculate minimum geometric dimensions required to deliver practical power output to a wireless water meter.
        Finally, we comment on the feasibility of using streaming cell technology as a means of harvesting energy from a domestic water supply.
    \end{abstract}

    \section{Introduction}
    Domestic and commercial water metering is becoming increasingly important throughout the world.
    Having a cheap and reliable way to retrieve metered data is an important consideration for utility companies.
    There are two practical methods of retrieving data from domestic water meters:
    \begin{enumerate}
        \item \emph{Visual inspection:}
            Involves a meter-reader visually sighting the meter's mechanical readout from time to time.
            Generally the reading frequency is kept to a minimum and estimate readings are made in the interim.
        \item \emph{Electronic telemetry:}
            Automatic wireless transmission of data to the billing company.
            Requires a power source at the site of metering, which is generally underground at a property's boundary.
        \end{enumerate}
    Due to the location of water metering, electrical power is usually provided by long life batteries.
    The batteries used in automatic meter readers are non-rechargeable and have a maximum life-span of 10 years.\cite{BMeters2014}
    Automatic meter reading enables utility companies to detect water leaks at consumers' properties.~\cite{Britton2013}

    The possibility of harvesting energy at the meter should allow frequent data reporting with less maintenance.
    Removing the batteries from current meter reading systems should reduce both the total cost of ownership and electrical waste.

    Streaming cells provide a way of converting fluid energy into electrical energy.
    They work on the principle that charged surfaces in contact with ionic solutions attract counter-ions\footnote{The terms counter-ion and co-ion refer to ions of opposite and like charge relative to the polarity of the solid surface respectively.} so as to form a charged layer at the contact surface.~\cite{Stein2004}
    This charged layer is termed the double layer and is the key mechanism that enables streaming cell energy conversion.
    It is possible to pump and collect ions from withing the double layer with the application of pressure.

    A double layer is called such because it is made of two inividual layers; the Stern and diffuse layers.
    % A double layer can be broken into two sub-layers; the Stern and diffuse layers.
    The Stern layer is comprised of immobilised ions situated very close the charged solid surface.~\cite{Salieb-Beugelaar2009}
    The diffuse layer is formed by the ions attracted to the surface, but not so strongly as to be immobile.
    A streaming cell moves ions in the diffuse layer through one or more narrow channels, concentrating them on the other side.

    Streaming cell channels can be created individually using a range fabrication methods, such as chemical etching or the use of narrowly separated parallel plates.
    Channels can also be formed en-mass by using porous materials such as glass or ceramics as the cell where the pores themselves form channels.
    When using glass, the surface spontaneously develops a negative surface charge when it comes in contact with water.
    The charge is caused by deprotonation at the contact surface (SiOH~$\leftrightarrows$~SiO$^{-}+$~H$^{+}$).
    By immersing a channel in an electrolyte solution double layers line the interior walls of the channel.
    The thickness of this layer is defined by the Debye length ($l_{D}$) and is controlled the ionic concentration of the liquid.~\cite{Israelachvili2011}

    When the dimensions of a channel become small the double layers begin to overlap and co-ions are repelled from the channel.
    Counter-ions within a streaming cell are collected by applying a pressure gradient across the channel/cell.
    Pressure forces the liquid through the channel (from which co-ions have been removed) creating a streaming current.
    It is the streaming current that raises the voltage (streaming potential) across the cell as the ions are pumped through it.

    Streaming currents have been heavily investigated as a means of generating electrical energy from pressure gradients.~\cite{Chang2009,Daiguji2006,Daiguji2004b,Davidson2008a,Davidson2008,CherngHon2012,Jiao2014,Lu2006,Olthuis2005,Osterle1964,Pennathur2007,Ren2008a,VanderHeyden2006,Heyden2007,Xie2008,Yang2003}
    % It is expected that a channel height equal to twice the Debye length leads to maximum energy conversion efficiency.~\cite{Daiguji2004b}
    Theoretical predictions of the efficiency of standard micro/nano-fluidic channels are 2\% for pure water and 7\% for sodium chloride.~\cite{VanderHeyden2006}
    Experimental results show conversion efficiencies in the range of:
    \begin{itemize}
        \item 0.01\% by forcing water through porous glass with pore sizes from 10\thinspace--\SI{16}{\micro\metre}.~\cite{Yang2003}
        \item 0.8\% by forcing pure water through a ceramic rod populated with \SI{6}{\micro\metre} pores.~\cite{Yang2004}
        \item 3\% by forcing sodium chloride solution through a \SI{75}{\nano\metre} by \SI{50}{\micro\metre} silica channel.~\cite{Heyden2007}
        \item 0.77\% by forcing sodium chloride through a \SI{200}{\nano\metre} pore in an alumina membrane.~\cite{Lu2006}
        \item 5\% by forcing sodium chloride solution through a \SI{0.5}{\nano\metre} cylindrical pore in polyethylene terephalate foil.~\cite{Xie2008}
    \end{itemize}
    These results indicate that solutions containing salt are more efficient as a working liquid, however the maximum efficiency is found when the salt concentration is low.~\cite{VanderHeyden2006}

    It is clear from the literature that there is significant progress to be made with respect to increasing the conversion efficiency.
    Techniques to induce hydrodynamic slip at the fluid-solid interface are predicted to increase this efficiency to 30-40\%.~\cite{Davidson2008a, Ren2008a}
    Experimental results utilising slip enhanced channels have not yet been reported in the literature.
    This type of enhancement would overcome the `no-slip boundary condition' at the solid-liquid interface.
    Slip at the interface boundary would mean that ions in the Stern layer, where ionic concentration is highest, could also be moved across the channel.
    Surface enhanced channels are out of the scope of this study due to cost and manufacturing difficulty.

    % In~\cite{Joly2006} it is suggested that a polarised hydrophobic surface would strongly enhance electro-osmotic effects and therefore further increase conversion efficiency

    We address the question of whether it is feasible to build a harvester using readily available materials.
    Such a harvester must produce enough energy to operate a microprocessor and radio communication device.
    It must collect all of its energy from a domestic water supply and be reliable.

    The remainder of this work is presented as follows.
    In section \ref{sect:streamingCell} we build and measure a simple streaming potential cell.
    We confirm the relationship between pressure and applied pressure and investigate the effect of channel height.
    In section \ref{sect:waterConsumption} we estimate the amount of harvestable energy from a constructed domestic water usage profile.
    Section \ref{sect:powerRequirements} quantifies energy requirements of a simple electronic meter.
    Following on this, section \ref{sect:harvesterSize} presents the required cell dimensions of a harvester capable of producing the requisite energy.
    In section \ref{sect:results} we combine and present results of our measurements and quantify our predictions.
    And finally, we conclude with a discussion on the feasibility of using streaming cells to harvest energy.

    \section{Readily Attainable Streaming Cell Power} \label{sect:streamingCell}
    In~\cite{Gu2000}, the authors employ a relatively simple parallel plate design to create a streaming cell.
    They glued plastic shims between parallel plates of glass, providing a simple way of controlling channel height.
    Using that method, the authors fabricated three cells with channel heights of \SI{50}{\micro\metre}, \SI{100}{\micro\metre} and \SI{150}{\micro\metre}.
    Each of the channels were \SI{3}{\centi\metre} long and had a inner width of \SI{1}{\centi\metre}.

    \begin{figure}
        \begin{center}
        \includegraphics[width=\linewidth]{Photo_streamingPotential_Assembly_Step3.JPG}
        \end{center}
        \caption{One of our ten constructed streaming cells mounted in an acrylic holder. Copper wires have been epoxied into each side to provide a means of measuring the developed potential across the cell. The plastic shims are visible as the orange strips at each side of the cell.}
        \label{fig:cell}
    \end{figure}

    We replicate the fabrication method and approximate dimensions of \cite{Gu2000} to produce ten streaming cells.
    The internal dimensions of which are \SI{3}{\centi\metre} long, \SI{1}{\centi\metre} wide and heights in the range of \SI{25}{\micro\metre} to \SI{250}{\micro\metre}.
    These dimensions are directly comparable to those of \cite{Gu2000}, where the length and width are the same but the heights span a wider range.
    The channels themselves are made from glass microscope slides (Sail Brand - JIA 7101WT) sectioned into halves.
    Plastic shims (Garlock Colorplast) are epoxied (Selleys Araldite Ultra Clear Resin) between slide halves to create the channels.
    The dimensions at each end of each channel are measured under a microscope to determine the actual channel height and ensure symmetry.

    Each channel is epoxied between two acrylic reservoirs and a base plate.
    This provides a rigid structure under pressure and provides connection of wires and hoses.
    Copper wires and silicon hoses are inserted into each reservoir to facilitate voltage measurement and the application of pressure.
    A photo of one of the cells is shown in Fig. \ref{fig:cell}.

    Pressure is applied across each cell by connecting one end to a tap and leaving the other open to atmosphere.
    A Honeywell pressure sensor (model 26PC15SMT) is connected across the input/output tubes to measure the developed pressure differential.
    An Agilent precision measurement mainframe (model E5270B) is used to measure streaming potential and read the pressure sensor's output.

    \section{Estimation of Harvestable Energy From A Typical New Zealand Dwelling}
    \label{sect:waterConsumption}
    \begin{figure}
        \begin{center}
        \includegraphics[width=\linewidth]{graph_profile.pdf}
        \end{center}
        \caption{Sample profile showing constructed instances of washing machine use, a shower and a toilet flush.
        Washing machine usage is broken into two parts corresponding to a wash and rinse cycle.}
        \label{fig:profileSample}
    \end{figure}

    \begin{figure}
        \begin{center}
        \includegraphics[width=\linewidth]{graph_pressureLoss.pdf}
        \end{center}
        \caption{Estimated head loss from a mechanical water meter typically installed in a domestic setting (Kent PSMT 25mm).}
        \label{fig:headloss}
    \end{figure}

    \begin{figure}
        \begin{center}
        \includegraphics[width=\linewidth]{graph_harvest.pdf}
        \end{center}
        \caption{Calculated power dissipation by a typical domestic mechanical water meter (Kent PSMT 25mm) for each of the sample profile events.}
        \label{fig:powerDissipated_meter}
    \end{figure}


    In \cite{Heinrich2008} the authors monitor water consumption of 51 homes throughout Auckland, New Zealand between February and September of 2008.
    The report shows that the majority of domestic water is consumed by the shower (30\%), washing machine (27\%) and toilet (20\%).
    Together these account for over three quarters of the indoor water usage in the average home.
    We have created a flow profile from which to calculate the available harvestable energy using data from both \cite{Heinrich2008} and \cite{Heinrich2007}.
    The profile fits the usage statistics of a home with two occupants according to the previously mentioned reports over the duration of a week.
    The during that time five uses of a washing machine, fourteen showers and fifty six toilet flushes occur.
    A sample of the usage profiles of each item is shown in Fig. \ref{fig:profileSample}.

    The Kent PSMT 25mm is a water meter typically installed in homes around the Auckland district.~\cite{WatercareNewZealand2014}.
    Figure~\ref{fig:headloss} shows the head loss curve that we have reconstructed based on product specifications for this meter.
    From that curve, we calculate the amount of power dissipated by the Kent PSMT meter during the average washing machine cycle, shower, and toilet flush.
    This data is presented as figure~\ref{fig:powerDissipated_meter}.
    The total energy dissipated within the water meter for each of the water events is:
    \begin{itemize}
    \item \SI{32.8}{\kilo\joule} per load of washing,
    \item \SI{13.3}{\kilo\joule} per shower, and
    \item \SI{1.45}{\kilo\joule} per flush of the toilet.
    \end{itemize}
    % These figures are based on average duration and flow rates as found in \cite{Heinrich2008}, and the estimated head loss from Fig.~\ref{fig:headloss}.
    Over a week, our reference water meter would dissipate a total of \SI{431}{\kilo\joule}, averaging \SI{61.6}{\kilo\joule} per day.
    
    \section{Estimation of Power Requirements}
    \label{sect:powerRequirements}

    
    \section{Estimation of Streaming Cell Harvester}
    \label{sect:harvesterSize}

    \section{Results}
    \label{sect:results}

    \section{Conclusion}
    \bibliographystyle{plain}
    \bibliography{library}

\end{document}
